% Options for packages loaded elsewhere
\PassOptionsToPackage{unicode}{hyperref}
\PassOptionsToPackage{hyphens}{url}
%
\documentclass[
]{article}
\usepackage{amsmath,amssymb}
\usepackage{lmodern}
\usepackage{iftex}
\ifPDFTeX
  \usepackage[T1]{fontenc}
  \usepackage[utf8]{inputenc}
  \usepackage{textcomp} % provide euro and other symbols
\else % if luatex or xetex
  \usepackage{unicode-math}
  \defaultfontfeatures{Scale=MatchLowercase}
  \defaultfontfeatures[\rmfamily]{Ligatures=TeX,Scale=1}
\fi
% Use upquote if available, for straight quotes in verbatim environments
\IfFileExists{upquote.sty}{\usepackage{upquote}}{}
\IfFileExists{microtype.sty}{% use microtype if available
  \usepackage[]{microtype}
  \UseMicrotypeSet[protrusion]{basicmath} % disable protrusion for tt fonts
}{}
\makeatletter
\@ifundefined{KOMAClassName}{% if non-KOMA class
  \IfFileExists{parskip.sty}{%
    \usepackage{parskip}
  }{% else
    \setlength{\parindent}{0pt}
    \setlength{\parskip}{6pt plus 2pt minus 1pt}}
}{% if KOMA class
  \KOMAoptions{parskip=half}}
\makeatother
\usepackage{xcolor}
\usepackage[margin=1in]{geometry}
\usepackage{graphicx}
\makeatletter
\def\maxwidth{\ifdim\Gin@nat@width>\linewidth\linewidth\else\Gin@nat@width\fi}
\def\maxheight{\ifdim\Gin@nat@height>\textheight\textheight\else\Gin@nat@height\fi}
\makeatother
% Scale images if necessary, so that they will not overflow the page
% margins by default, and it is still possible to overwrite the defaults
% using explicit options in \includegraphics[width, height, ...]{}
\setkeys{Gin}{width=\maxwidth,height=\maxheight,keepaspectratio}
% Set default figure placement to htbp
\makeatletter
\def\fps@figure{htbp}
\makeatother
\setlength{\emergencystretch}{3em} % prevent overfull lines
\providecommand{\tightlist}{%
  \setlength{\itemsep}{0pt}\setlength{\parskip}{0pt}}
\setcounter{secnumdepth}{-\maxdimen} % remove section numbering
\ifLuaTeX
  \usepackage{selnolig}  % disable illegal ligatures
\fi
\IfFileExists{bookmark.sty}{\usepackage{bookmark}}{\usepackage{hyperref}}
\IfFileExists{xurl.sty}{\usepackage{xurl}}{} % add URL line breaks if available
\urlstyle{same} % disable monospaced font for URLs
\hypersetup{
  pdftitle={Assignment-5 },
  pdfauthor={Imanbayeva Sofya, Mazzi Lapo, Piras Mattia, Srivastava Dev },
  hidelinks,
  pdfcreator={LaTeX via pandoc}}

\title{Assignment-5 \vspace{1in}}
\usepackage{etoolbox}
\makeatletter
\providecommand{\subtitle}[1]{% add subtitle to \maketitle
  \apptocmd{\@title}{\par {\large #1 \par}}{}{}
}
\makeatother
\subtitle{Group 22 \vspace{1in}}
\author{Imanbayeva Sofya, Mazzi Lapo, Piras Mattia, Srivastava Dev
\vspace{1in}}
\date{2023-04-26 \vspace{1in}}

\begin{document}
\maketitle

\newpage

\hypertarget{question-1---filtered-estimates}{%
\section{Question 1 - Filtered
Estimates}\label{question-1---filtered-estimates}}

The following plot shows the annual flow in the river Nile, the data we
are going to model using a dynamic linear model

\includegraphics[width=0.75\linewidth,height=0.75\textheight]{assignment5_files/figure-latex/Nile graph-1}

Let us consider the following random walk plus noise model to be applied
to this Nile data: \begin{eqnarray*} 
Y_t &= \theta_t + v_t \quad    & v_t \overset{i.i.d.}\sim N(0, V)\\
\theta_t &= \theta_{t-1} + w_t \quad   & v_t \overset{i.i.d.}\sim N(0, W)
\end{eqnarray*}

We will set V = 15100 and W = 1470 and the initial distribution
\(\theta_0 \sim N(1000,1000)\) for our model

Plotting the filtered estimates we get

\includegraphics[width=0.75\linewidth,height=200px]{assignment5_files/figure-latex/plot filtered estimates-1}

We can compute the variance as well. It is plotted as follows

\includegraphics[width=0.75\linewidth,height=200px]{assignment5_files/figure-latex/variance compute-1}

As we can see from the graph above, the standard deviations of the
filtering estimates decrease as time passes by, and converges to a value
above 60. This might reflect the low guess of the standard deviation C0
(sqrt(1000) = 31) made at the beginning . As new observations arrive the
standard deviation converges to the stable value.

We finally plot the whole data along with the filtered estimates with
95\% confidence interval

\includegraphics[width=0.75\linewidth,height=0.75\textheight]{assignment5_files/figure-latex/final plot-1}

\hypertarget{question-2---one-step-ahead-forecasts}{%
\section{Question 2 - One-step ahead
forecasts}\label{question-2---one-step-ahead-forecasts}}

Below is the plot for the one-step ahead forecast with 95\% confidence
intervals

\includegraphics[width=0.75\linewidth,height=0.75\textheight]{assignment5_files/figure-latex/forecasting-1}

\hypertarget{question-3---signal-to-noise-ratio}{%
\section{Question 3 - Signal to noise
ratio}\label{question-3---signal-to-noise-ratio}}

The signal-to-noise ratio are an important factor which set the weight
of the most recent data point for the filtered estimates, with a larger
value putting more weight to the recent data-point.

To test the effect of signal-to-noise ratio on our model we test the
following models. \linebreak

Model 1 (Signal-to-noise ratio = 0.097) - \begin{eqnarray*} 
Y_t &= \theta_t + v_t \quad    & v_t \overset{i.i.d.}\sim N(0, 15100)\\
\theta_t &= \theta_{t-1} + w_t \quad   & v_t \overset{i.i.d.}\sim N(0, 1470)
\end{eqnarray*}

Model 2 (Signal-to-noise ratio = 10.27)- \begin{eqnarray*} 
Y_t &= \theta_t + v_t \quad    & v_t \overset{i.i.d.}\sim N(0, 1470)\\
\theta_t &= \theta_{t-1} + w_t \quad   & v_t \overset{i.i.d.}\sim N(0, 15100)
\end{eqnarray*}

Plotting the filtered estimates of both the models we find

\includegraphics[width=0.75\linewidth,height=0.75\textheight]{assignment5_files/figure-latex/signal to noise graph-1}

\linebreak

We can see from the graph that when we increase the signal to noise
ratio (W/V) substantially, the filtered estimates follow the data more
closely. This can be explained, because when we increase W/V we increase
the weight on the most recent data point. In the plot, given we increase
the ratio hundredfold the estimates almost match the most recent points.

\end{document}
